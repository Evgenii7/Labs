
\begin{center}
\textbf{Лабораторная работа №5}
\\
\textbf{Тема:} Работа с пакетным менеджером для С++.
\\
Цель:В ходе лабораторной работы необходимо научится рарабатывать с пакетным менеджером С++.  
\end{center}

Требуемое ПО:
\begin{enumerate}

\item Visual Studio 15.6.3 и выше с установленным пакетом для разработки С++.\
\item CMake 3.11 - https://cmake.org/download
\item git - https://git-scm.com/
\item cppan - https://cppan.org/client/
\item при установке git, cmake выбрать пункт - Добавить в PATH.
\item cppan.exe скопировать в C:\Program Files\CMake\bin\\
\end{enumerate}

Порядок выполнения:
\\
\begin{enumerate}

\item Запустить VS
\item Создать проект С++ -> CMake. При создание проекта снять флажок “Создать папку для проекта”.
\item Выполнить сборку созданного проекта
\item Открыть главный CMakeLists.txt (в самой верхней папке)
\item До строки add\_executable добавить\\
find\_package(CPPAN REQUIRED)\\
cppan\_add\_package(\\
	pvt.cppan.demo.intel.opencv.highgui-3\\
)\\
cppan\_execute()\\

\item После строки add\_executable добавить\\
target\_link\_libraries(ПервыйАргументИзAddExecutable\\
	pvt.cppan.demo.intel.opencv.highgui\\
)\\

\end{enumerate}

Код программы CMakeLists.txt:\\

\# CMakeList.txt: проект CMake для CMakeProject1; включите исходный код и определения\\
\# укажите здесь логику для конкретного проекта.\\
\#\\
cmake\_minimum\_required (VERSION 3.8)\\
\\
\# Добавьте источник для исполняемого файла этого проекта.\\
find\_package(CPPAN REQUIRED)\\
cppan\_add\_package(\\
	pvt.cppan.demo.intel.opencv.highgui-3\\
)\\
cppan\_execute()\\
\\
add\_executable (CMakeProject1 "CMakeProject1.cpp" "CMakeProject1.h")\\
target\_link\_libraries(CMakeProject1 pvt.cppan.demo.intel.opencv.highgui)\\
\\
\\
\# TODO: Добавьте тесты и целевые объекты, если это необходимо.\\

Код программы CMakeProject1\\

// CMakeProject1.cpp: определяет точку входа для приложения.\\
//\\
\\
\#include <opencv2/highgui.hpp>\\
\#include "CMakeProject1.h"\\
\\
using namespace std;\\
\\
int main()\\
{\\
	auto i = cv::imread("D:\\КОПИЯ ФЛЭШКИ\\4 СЕМЕСТР\\Инструментальные средства разработки программного  обеспечения Пугин Е.В\\L55\\fot\\f.jpg");\\
	i = 255 - i; // инверсия изображения\\
	cv::imwrite("D:\\КОПИЯ ФЛЭШКИ\\4 СЕМЕСТР\\Инструментальные средства разработки программного\\ обеспечения Пугин Е.В\\L55\\fot\\fo.bmp", i);\\
\\
	cout << "Hello CMake." << endl;\\
	return 0;\\
}\\

\begin{figure}[h]% добавляем рисунок.
\centering
\includegraphics[scale=0.6]{1-1}
\caption{Результат работы программы}
\label{fig:1-1}
\end{figure}




\newpage
\textbf{Вывод:}
В ходе лаборпторной работы я научился работать с пакутным менеджером С++.




